
\section{\acs{DDL} Commands In Depth}
\paragraph{} In this section we will cover the \acf{DDL} commands. These commands deal with the structure of the database itself, like create tables, views, procedures etc. or modify/delete them.\\\textbf{Purpose:} Define the structure of the database and its objects.\\\textbf{Auto-Commit:} Yes. Once a \acs{DDL} command is executed changes will be permanent, no rollback option.
\subsection{Schema Management}
\paragraph{} Schema is a collection of database objects, like tables, views, triggers, etc.
\begin{lstlisting}[language=SQL]
	-- Create a database schema
	-- Syntax: create database <schema_name>;
	-- Example:
	create database customer;
	
	-- View list of all existing schemas (current user has access to)
	show databases;
	
	-- View currently active/in use schema
	select database();
	
	-- select active schema for use
	-- Syntax: use <schema_name>;
	-- Example:
	use customer;
	
	-- Delete database schema
	-- Syntax: drop database <schema_name>;
	-- Example:
	drop database customer;
\end{lstlisting}
\subsection{Table Management}
\paragraph{} Table is the base component of relational databases, it's the place that data are stored based on the defined structure.
\begin{lstlisting}[language=SQL]
	use booksdb;
	select database();
	show tables;
	
	-- Create new table
	-- Syntax: create table <table_name> (column1 datatype, column2 datatype, column3 datatype ....);
	-- Example:
	create table customer (
	customer_id bigint primary key auto_increment,
	first_name varchar(50) not null,
	last_name varchar(50),
	birthdate date,
	age int,
	constraint is_adult_failed check (age > 18)
	);
	create table products (
	product_id bigint auto_increment,
	pname varchar(10) not null,
	pstatus char(1) default 'E',
	customer_id bigint,
	primary key (product_id),
	foreign key (customer_id) references customer (customer_id)
	);
	-- Some notes:
	-- 1. Table product references table customer, so customer has to be created first.
	-- 2. In the two tables we see the two different ways of defining the primary key; the primary key may consist of more than one column.
	-- 3. Primary keys cannot be null.
	-- 4. not null vs default: not null means during insert/update you have to give any non null valid value for that column. Default means, if no value is passed during insert for that column, default value will be used. However during insert or update, null is a valid value to set for the column.
	-- 5. char vs varchar: char means fixed size in storage, if we set char(10) for example, even if we give as value the string 'Hi', the space used in storage will be for 10 chars. With varchar(10) however the size is not fixed, depending on the size of the string, the required space will be used. In both cases, string longer than the given value will cause failure or data to be truncated.
	-- 6. We can add check constraints named or unnamed to prevent invalid data from being inserted in the database.
	-- 7. Foreign key can be null, if we don't explicitly say not null, however if value is given, that value must exist in the reference table.
	-- 8. auto_increment means if no value is passed, it will take the max value of that table column + 1.
\end{lstlisting}
\begin{itemize}
	\item For more on create tables \href{https://dev.mysql.com/doc/refman/8.0/en/create-table.html}{click here}.
	\item For more on data types \href{https://dev.mysql.com/doc/refman/8.0/en/data-types.html}{click here}.
\end{itemize}
\subsection{Practice: Understand Created Tables}
\paragraph{} The commands in this section are not \acs{DDL}, they are \acs{DML} and \acs{DQL}, just to see in practice how primary keys, foreign keys, check constraints etc work to get a better understanding.
\begin{lstlisting}[language=SQL]
	-- Tables are initially empty
	select * from customer;
	select * from products;
	
	-- customer table
	insert into customer 
	(customer_id, first_name, last_name, birthdate, age)
	values
	(1, 'Spyros', null, null, null)
	; -- will work
	insert into customer 
	(customer_id, first_name, last_name, birthdate, age)
	values
	(2, 'John', 'Wick', null, 15)
	; -- will fail due to age constraint validation
	insert into customer 
	(customer_id, first_name, last_name, birthdate, age)
	values
	(2, 'John', 'Wick', null, 45)
	; -- will work
	insert into customer 
	(first_name, last_name, birthdate, age)
	values
	('Bruce', 'Willis', null, 45)
	; -- will work (we removed customer id but will be auto populated due to auto_increment attribute)
	insert into customer 
	(customer_id, first_name, last_name, birthdate, age)
	values
	(3, 'Nick', null, null, 45)
	; -- will fail due to duplicate primary key
	select * from customer;
	
	-- products table
	insert into products
	(product_id, pname, pstatus, customer_id)
	values
	(3, 'R7', null, 4)
	; -- will fail because we gave inexistent foreign key
	insert into products
	(product_id, pname, pstatus, customer_id)
	values
	(2, 'Ryzen 7 1700', null, 1)
	; -- will fail because data is too long for column pname
	insert into products
	(product_id, pname, pstatus, customer_id)
	values
	(3, 'R7 ', null, 1)
	; -- will work
	insert into products
	(product_id, pname, pstatus, customer_id)
	values
	(5, 'R8 ', 'D', 3)
	; -- will work
	insert into products
	(pname, customer_id)
	values
	('R9 ', 2)
	; -- will work
	
	delete from customer where customer_id = '1'; -- will fail because there are data in table products that reference that customer id
	
	select * from customer;
	select * from products;
\end{lstlisting}
\subsection{More on Table Management}
\begin{lstlisting}[language=SQL]
	-- Create table with select from another table - using where clause we can only keep data that satisfy certain conditions
	-- Syntax: create table <backup_table_name> select * from <original_table_name>;
	-- Example:
	create table customer_backup_26082023 select * from customer;
	
	select * from customer_backup_26082023;
	
	-- Create table with same structure of another table but without data
	-- Syntax: create table <clone_table_name> select * from <original_table_name> where 1 = 2;
	-- Example:
	create table products_clone select * from products where 1 = 2; -- 1 = 2 will never be true and is used to insert no data
	
	select * from products_clone; -- no data
	
	desc products;
	desc products_clone; -- notice that PK/FK/auto_increment/check constraints etc are not copied on the new table, only the structure.
	
	-- Rename table
	-- Syntax: rename table <original_table_name> to <new_table_name>;
	-- Example:
	rename table products to product;
	
	select * from products; -- no table with name products now
	select * from product;
	
	-- Alternatively alter can be used
	-- Syntax: alter table <original_table_name> to <new_table_name>;
	-- Example:
	alter table products_clone rename products_clone2;
	
	select * from products_clone; -- no table with this name now
	select * from products_clone2;
	
	-- Alter table
	-- Alter table can be used to modify the attributes of existing tables, viewe or any database object
	-- Depending on the DBMS and its settings, what is allowed and what is not allowed to alter might slightly differ
	
	-- Remove column from table
	-- Syntax: alter table <table_name> drop <column_name>;
	alter table customer drop age;
	-- confirm column is removed
	desc customer;
	-- Add column to table
	-- Syntax: alter table <table_name> add <column_name>;
	alter table customer add age int constraint is_adult_failed check (age > 18);
	-- confirm column is added back
	desc customer;
	
	-- Truncate table
	-- Truncate drops and creates again the table but without data
	select * from customer_backup_26082023; -- table has data before truncate
	
	-- Syntax: truncate <table_name>;
	-- Example:
	truncate customer_backup_26082023;
	
	select * from customer_backup_26082023; -- no data after truncate but table still exists
	
	-- Alternatively - However this is wrong way to delete all data from a table
	-- Syntax: delete from <table_name>;
	delete from customer_backup_26082023; -- without where clause it will delete all data from the table
	
	-- View table details
	-- Syntax A: describe <table_name>;
	-- Syntax B: desc <table_name>;
	-- Example:
	desc customer;
	
	-- Drop table
	-- Syntax: drop table <table_name>;
	-- Example:
	drop table customer_backup_26082023;
	drop table products_clone2;
	
	-- Truncate vs Delete vs Drop
	-- Truncate: Truncate is used to delete all the data from a table but keep the table itself.
	-- Truncate will drop and create again the table.
	-- Delete: without where clause will also empty the table. However because delete will have to go through the whole
	-- table and delete each row individually, which is a lot slower and resource intensive than truncate.
	-- Drop: Completely deletes the table and all its data.
	
	-- Clean-up DB
	drop table product;
	drop table customer;
\end{lstlisting}
\begin{itemize}
	\item Fore more on alter tables \href{https://dev.mysql.com/doc/refman/8.0/en/alter-table.html}{click here}.
\end{itemize}
\subsection{Views}
\paragraph{} Views act like virtual tables to give as an opinionated view of the data. Depending on how the view is created it may or may not allow insert/update/delete operations. However if these operations are allowed, the changes will be done on the actual tables and not only on the view.
\begin{lstlisting}[language=SQL]
	select	c.customer_id, c.first_name, c.last_name, p.product_id, p.pname, p.pstatus
	from	customer as c, product as p
	where	c.customer_id = p.customer_id;
	-- With above query we get all customers and their products and with specified columns
	-- Products without customer (if any exist) won't be in the result set
	
	-- Create view
	-- Syntax: create view <viewname> as select <columnnameA>, <columnnameB>, .... from <tablenamea>, <tablenameb>, .... WHERE <conditions>;
	-- Example:
	create view customer_products as
	select	c.customer_id, c.first_name, c.last_name, p.product_id, p.pname, p.pstatus
	from	customer as c, product as p
	where	c.customer_id = p.customer_id
	;
	
	-- Now instead of previous big query we can just do:
	select * from customer_products; -- and get same results
	
	-- Update view
	-- Syntax: create or replace view <viewname> as select <columnnameA>, <columnnameB>, .... from <tablenamea>, <tablenameb>, .... WHERE <conditions>;
	-- Example:
	create or replace view customer_products as
	select	c.customer_id, c.first_name, c.last_name, c.age, p.product_id, p.pname, p.pstatus
	from	customer as c, product as p
	where	c.customer_id = p.customer_id
	;
	-- create or replace, if view exists it will change it (if there is anything different) or create it if it doesn't exist
	
	desc customer_products;
	
	-- Drop view
	-- Syntax: drop view <viewname>;
	-- Example:
	drop view customer_products;
	
	-- Clean-up DB
	drop table product;
	drop table customer;
\end{lstlisting}
\begin{itemize}
	\item Fore more on views \href{https://dev.mysql.com/doc/refman/8.0/en/views.html}{click here}.
	\item Fore more on updatable/insertable views \href{https://dev.mysql.com/doc/refman/8.0/en/view-updatability.html}{click here}.
\end{itemize}
\subsection{Section Notes}
\begin{itemize}
	\item For all \acs{SQL}/\acs{DDL} commands of this section \href{file:./source-items/sql/2-sql-ddl.sql}{click here}.
\end{itemize}
