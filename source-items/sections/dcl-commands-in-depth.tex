
\section{\acs{DCL} Commands In Depth}
\paragraph{} In this section we will cover the \acf{DCL} commands and everything related to user and role management.\\\textbf{Purpose:} deal with the rights and permissions of the users on the database.\\\textbf{Users:} are user accounts that can connect in the \acs{DB} to perform their tasks.\\\textbf{Roles:} are like predefined sets of privileges that can be granted to users.\\Connect with MySQL Workbench as root or admin user to execute the following commands.\\\textbf{Auto-Commit:} Yes. Once a \acs{DCL} command is executed changes will be permanent, no rollback option.
\subsection{User Management}
\paragraph{} \textbf{Note:} These are not \acs{DCL} commands but we need them to see the content in a linear way. These commands are all auto-committed, once executed changes are saved in \acs{DB} without the ability to do rollback.
\begin{lstlisting}[language=SQL]
	-- Create new user
	-- Syntax: create user '<username>'@'<hostname>' identified by '<password>';
	-- Example:
	create user 'testuser'@'localhost' identified by 'testpass';
	-- Note: By default new users have no privileges. They have to be granted their privileges seperately.
	
	-- View current users
	-- MySQL Syntax:
	select * from mysql.user;
	
	-- Rename user
	-- Syntax: rename user '<oldusername>'@'<hostname>' to '<newusername>'@'<hostname>';
	-- Example:
	rename user 'testuser'@'localhost' to 'realuser'@'localhost';
	
	-- Delete user
	-- Syntax: drop user '<username>'@'<hostname>';
	-- Example:
	drop user 'realuser'@'localhost';
\end{lstlisting}
\subsection{Grant/Revoke Privileges}
\begin{lstlisting}[language=SQL]
	-- View admin level privileges of users (like create/drop databases, users etc)
	-- NySQL Syntax:
	select * from mysql.user;
	
	-- Create testuser to experiment with
	create user 'testuser'@'localhost' identified by 'testpass';
	-- create testdb to expirement with
	create database testdb;
	
	-- See user privileges on databasese and tables
	-- Syntax: show grants for '<username>'@'<hostname>';
	-- Example:
	show grants for 'admin'@'%';
	show grants for 'testuser'@'localhost';
	
	-- View list of all available privileges
	show privileges;
	
	-- Alter user privileges
	-- Give all permissions for specific database
	-- Syntax: grant all privileges on <db_name>.* TO '<username>'@'<hostname>';
	-- Example
	grant all privileges on testdb.* TO 'testuser'@'localhost';
	-- * can be replaced with table name, in that case privileges will be granted on table level and not on db
	
	-- Remove all permissions for specific database
	-- Syntax: revoke all privileges on <db_name>.* from '<username>'@'<hostname>';
	-- Example:
	revoke all privileges on testdb.* from 'testuser'@'localhost';
	
	-- Give specific permission(s) only 
	-- Syntax: grant <privilegeA>, <privilegeB>, <....> on <db_name>.* to '<username>'@'<hostname>';
	-- Example - Give only insert, select permissions on all tables for DB:
	grant insert, select on testdb.* to 'testuser'@'localhost';
	
	-- Remove specific permission(s) only
	-- Syntax: revoke <privilegeA>, <privilegeB>, <....> on <db_name>.* from '<username>'@'<hostname>';
	-- Example - Revoke only INSERT permission from user:
	revoke insert on testdb.* from 'testuser'@'localhost';
	
	-- Flush privileges is required to make the changes effective, only when we
	-- directly modify the privileges in the grant tables using insert, update, delete.
	-- Syntax:
	flush privileges;
	
	-- Note: Privilege changes may not be reflected on existing DB sessions.
\end{lstlisting}
\subsection{Role Management}
\paragraph{} The purpose of roles is to easily grant specific set of privileges to users based on the role we want them to have on the database.
\begin{lstlisting}[language=SQL]
	-- Create new role
	-- Syntax: create role <rolename>;
	-- Example:
	create role 'testrole';
	
	-- View existing roles in DB
	-- MySQL Syntax:
	select * from mysql.user where authentication_string = ''; -- only roles should have empty authentication string
	
	-- Rename Role:
	-- Syntax: rename user <oldrolename> to <newrolename>;
	-- Example:
	rename user 'testrole' to 'mytestrole';
	
	-- Add privileges to role
	-- Syntax: grant all privileges on <db_name>.* to 'role';
	-- Example:
	grant all privileges on testdb.* to 'mytestrole';
	
	-- Remove privileges from role
	-- Syntax: revoke all privileges on <db_name>.* from 'role';
	-- Example:
	revoke all privileges on testdb.* from 'mytestrole';
	
	show grants for 'mytestrole';
	
	-- Add role to user
	-- Syntax: grant <role> to '<user>'@'<host>';
	-- Example:
	grant 'mytestrole' to 'testuser'@'localhost';
	
	-- Remove role from user
	-- Syntax: revoke <role> from '<user>'@'<host>';
	-- Example:
	revoke 'mytestrole' from 'testuser'@'localhost';
	
	show grants for 'testuser'@'localhost';
	
	-- Set default role for user
	-- Syntax: set default role '<role>' to '<user>'@'<host>';
	-- Example:
	set default role 'mytestrole' to 'testuser'@'localhost';
	
	-- Remove default role from user
	-- Syntax: set default role none to '<user>'@'<host>';
	-- Example:
	set default role none to 'testuser'@'localhost';
	
	-- View current users role
	-- Syntax:
	select current_role();
	
	-- Delete role
	-- Syntax: drop role <rolename>;
	-- Example:
	drop role 'mytestrole';
	
	drop user 'testuser'@'localhost';
	drop database testdb;
\end{lstlisting}
\subsection{Initialize Sample Database}
\paragraph{} At this point please run these two scripts below as admin to create the database and insert sample data.\\
1-SQL Script to create tables: \href{file:./source-items/sql/books-db-init/booksdb-create-tables.sql}{click here}.\\
2-SQL Script to insert data: \href{file:./source-items/sql/books-db-init/booksdb-insert-data.sql}{click here}.
\subsection{Practice: Role and User Management}
\begin{lstlisting}[language=SQL]
	select * from mysql.user;
	
	create user 'bookso'@'localhost' identified by 'bookso';
	show grants for 'bookso'@'localhost';
	
	-- Open another session (let's name it session B) as bookso user now and run
	show databases; -- database booksdb won't be visible
	
	grant all privileges on booksdb.* to 'bookso'@'localhost';
	
	-- On Session B as bookso user run again
	show databases; -- database booksdb will be visible and with grant all we are able to do any operation on the db
	
	revoke all privileges on booksdb.* from 'bookso'@'localhost';
	
	create role 'books_read_role';
	create role 'books_insert_role';
	create role 'books_delete_role';
	
	select * from mysql.user where authentication_string = '';
	
	show grants for 'books_read_role';
	grant select on booksdb.* to 'books_read_role';
	
	show grants for 'books_insert_role';
	grant insert on booksdb.* to 'books_insert_role';
	
	show grants for 'books_delete_role';
	grant delete on booksdb.* to 'books_delete_role';
	
	grant 'books_read_role' to 'bookso'@'localhost';
	grant 'books_insert_role' to 'bookso'@'localhost';
	grant 'books_delete_role' to 'bookso'@'localhost';
	show grants for 'bookso'@'localhost';
	
	-- Note: By default roles are not active on users, this means users won't have these privileges
	-- even if they are granted these privileges.
	
	-- Open a new session as a bookso user and run below commands
	show databases; -- booksdb database won't be visible
	-- Solution 1 (Per Session Only):
	select current_role(); -- will be none
	set role 'books_read_role'; -- after this command above command will show new role
	-- You can also asign a list of roles:
	set role 'books_read_role', 'books_insert_role';
	
	-- This way the user has the privileges of the set roles only.
	-- With set role, each time you open a new session you will have to set again each role.
	
	-- Solution 2 (Permanent):
	set default role 'books_insert_role', 'books_read_role' to 'bookso'@'localhost';
	
	-- Now open a new session as bookso and run below command, all above roles will be listed
	select current_role(); -- will show all 3 above roles
	
	-- Note: If you set the role for the user of current user the part to 'user'@'host' is optional
	-- but if you set it for another user it's mandatory
	
	-- Note: Such changes won't be reflected to existing sessions, only on new ones (set default role)
	
	-- Clean up
	drop role 'books_insert_role', 'books_read_role', 'books_delete_role';
	grant all privileges on booksdb.* to 'bookso'@'localhost';
\end{lstlisting}
\subsection{Section Notes}
\begin{itemize}
	\item For all \acs{SQL}/\acs{DCL} commands of this section \href{file:./source-items/sql/1-sql-dcl.sql}{click here}.
	\item For more on account management statements \href{https://dev.mysql.com/doc/refman/8.0/en/account-management-statements.html}{click here}.
\end{itemize}
