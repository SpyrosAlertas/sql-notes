
\section{\acs{DML} Commands In Depth}
\paragraph{} In this section we will cover the \acf{DML} commands which allow us to modify the data in the database. From \acs{CRUD}, these commands are responsible for Create/Update/Delete operations.\\\textbf{Purpose:} Modify data in the database.\\\textbf{Auto-Commit:} No. Some \acs{DBMS} auto-commit \acs{DML} commands as well.
\subsection{Preparation}
\begin{lstlisting}[language=SQL]
	-- Note: Insert/Update/Delete are DDL commands; start transaction/commit/rollback are TCL commands
	
	-- Note: In MySQL by default only safe update/delete is allowed, which means only by using a primary key in where clause. To change this please run below two commands:
	set global sql_safe_updates = 0; -- disable safe updates for all sessions
	set sql_safe_updates = 0; -- disable safe updates for current session only
	
	show global variables like 'sql_safe_updates';
	show variables like 'sql_safe_updates';
	
	-- This global variable is ignored in MySQL Workbench unless you change below setting:
	-- MySQL Workbench -> Edit -> Preferences -> SQL Editor -> Uncheck Safe Updates box at the end
	
	-- Note: In MySQL even DML commands are autocommited. There are two methods to work around this:
	-- Method 1: Disable auto-commit variable to 0
	set global autocommit = 0; -- disable autocommit for all sessions
	set autocommit = 0; -- disable autocommit for current session only
	
	show global variables like 'autocommit';
	show variables like 'autocommit';
	
	-- This global variable is also ignored unless you change below setting:
	-- MySQL Workbench -> Edit -> Preferences -> SQL Execution -> Uncheck New connections use auto commit mode
	
	-- Method 2: Start a transaction before executing DML commands
	start transaction;
	-- run DML commands - changes won't saved untill you commit or rollback
	commit;
	-- or
	rollback;
	-- to complete the transaction by saving or reverting the changes
	
	-- Note: If you commit or rollback, then you have to start another transaction, else DML commands will auto-commit again
	-- Note: Each time start transaction; is executed, pending changes in current session will be committed
\end{lstlisting}
\subsection{Insert}
\begin{lstlisting}[language=SQL]
	use booksdb;
	
	desc book;
	
	-- Insert
	-- Syntax: insert into <table_name>
	-- Example:
	start transaction;
	insert into book
	(title, release_year, book_language)
	values
	('New Experimental Cook Book', 2004, 'English')
	;
	commit;
	rollback; -- if you rollback before commit, change will be lost, if you rollback after commit, change is permanent already
	select * from book where title = 'New Experimental Cook Book';
	
	-- If you give all values in same order as they appear in table, syntax can change as below (skipping column names)
	-- Syntax:
	-- insert into book (value1, value2, ....);
	
	-- Insert using select (can also be from different table)
	-- Syntax: insert into <table_name> <columnA, columnB, .... > select <columnA, columnB> from <table_name> where <conditions>;
	-- Example:
	start transaction;
	insert into book
	(title, release_year, book_language)
	select 'Animal Farm v2', release_year + 5, book_language
	from book
	where title = 'Animal Farm'
	;
	commit;
	select * from book where title = 'Animal Farm';
	select * from book where title = 'Animal Farm v2';
\end{lstlisting}
\subsection{Update}
\begin{lstlisting}[language=SQL]
	-- Hint: Before update it's always a good idea to run a select query, to check what the where clause will return. If the were clause is not correct, undesired rows may be updated, if you do select first you can confirm it works as should. Same advice goes for delete.
	
	select * from book where title = 'New Experimental Cook Book'; -- 1 row
	
	-- Update
	-- Syntax: update <table_name> set <columnA = 'valueA', columnB = 'valueB', ....> where <conditions>;
	-- Example:
	start transaction;
	update book
	set release_year = 2006
	where title = 'New Experimental Cook Book'; -- 1 row updated
	commit; -- if more than expected rows are updated rollback and check
\end{lstlisting}
\subsection{Delete}
\begin{lstlisting}[language=SQL]
	select * from book where title = 'Animal Farm v2'; -- 1 row
	select * from book where title = 'New Experimental Cook Book'; -- 1 row
	
	-- Syntax: delete from <table_name> where <conditions>;
	-- Example:
	start transaction;
	delete from book
	where title = 'New Experimental Cook Book'; -- 1 row deleted
	delete from book
	w`here title = 'Animal Farm v2'; -- 1 row deleted
	commit;
\end{lstlisting}
\subsection{Section Notes}
\begin{itemize}
	\item For more on insert \href{https://dev.mysql.com/doc/refman/8.0/en/insert.html}{click here}.
	\item For more on update \href{https://dev.mysql.com/doc/refman/8.0/en/update.html}{click here}.
	\item For more on delete \href{https://dev.mysql.com/doc/refman/8.0/en/delete.html}{click here}.
	\item For all \acs{SQL}/\acs{DML} commands of this section \href{file:./source-items/sql/3-sql-dml.sql}{click here}.
\end{itemize}
