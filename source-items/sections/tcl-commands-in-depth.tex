
\section{\acs{TCL} Commands In Depth}
\paragraph{} In this section we will cover the \acf{TCL} commands. These commands deal with the behavior of transactions.\\\textbf{Purpose:} Define the behavior of transactions.\\\textbf{Auto-Commit:} -.
\subsection{Commit/Rollback}
\begin{lstlisting}[language=SQL]
	commit; -- data changes made via DML commands in current session will be made permanent
	rollback; -- non committed data changes made via DML commands in current session will be reverted
	
	-- Note: commit/rollback have no impact on what is done in other sessions
\end{lstlisting}
\subsection{Implicit Commit}
\paragraph{} In previous section we saw the commit and rollback commands, which explicitly save or revert changes done by \acs{DML} commands. However, there are other \acs{SQL} commands, that if run, they will implicitly end any pending transaction in session and silently commit unsaved changes.
\begin{itemize}
	\item \acs{DDL} commands
	\item Some \acs{TCL} commands (Start Transaction/Begin Work/Begin)
	\item General Rule: Any command that auto-commits.
\end{itemize}
\paragraph{} So it's advised, if there are uncommitted changes, before running \acs{SQL} commands other than \acs{DML} and \acs{DQL}, either (1) commit/rollback or (2) open a new session and execute these commands there to avoid accidentally committing unwanted changes.
\begin{lstlisting}[language=SQL]
	-- Example of Implicit Commit
	
	select * from book where title = 'Experimental Cooking Recipes'; -- no such book initially
	
	start transaction;
	insert into book (title, release_year, book_language) values ('Experimental Cooking Recipes', 2004, 'English');
	-- if you check before rollback book will be there, run rollback and check again, book won't be there anymore
	rollback;
	
	-- now we will try again the same as above, but before rollback we will run DQL command create table
	
	start transaction;
	insert into book (title, release_year, book_language) values ('Experimental Cooking Recipes', 2004, 'English');
	create table tmp (tmp_id int primary key, value varchar(10)); -- now before rollback run this DQL command
	rollback;
	
	-- now book remains even after rollback - this is because DQL commands like create/drop table autocommit
	
	drop table tmp; -- we no longed need this table
\end{lstlisting}
\subsection{Start Transaction/Begin Work/Begin}
\paragraph{} One thing to note is that in some \acs{DBMS} start transaction is silently executed each time any \acs{DML} command is executed (like in Oracle SQL) but in other ones like MySQL this does not happen (by default) unless you explicitly start a transaction, else any \acs{DML} command is considered as a single transaction that will be auto-committed once executed. Solutions in systems like MySQL:
\begin{itemize}
	\item Explicitly start a transaction each time
	\item Disable auto-commit session variable (set autocommit = 0;)
	\item Disable auto-commit global variable (set global autocommit = 0;) and start a new session
	\subitem Note: In MySQL Workbench you also have to change the below setting:
	\subitem Edit -> Preferences -> SQL Execution -> Uncheck option: New connections use auto commit mode
	\subitem Otherwise global variable autocommit is ignored
\end{itemize}
\paragraph{} We will work by explicitly starting a new transaction each time.
\begin{lstlisting}[language=SQL]
	-- start transaction = begin work = begin. They do the same thing, start a new transaction
	-- running these commands will disable auto-commit setting for DML commands if it's enabled, until transaction is completed (by commit/rollback or implicit commit)
	
	start transaction;
	-- execute DQL/DML commands
	commit; -- or rollback;
	
	-- or
	begin work;
	-- execute DQL/DML commands
	commit; -- or rollback;
	
	-- or
	begin;
	-- execute DQL/DML commands
	commit; -- or rollback;
\end{lstlisting}
\subsection{Rollback to Savepoint}
\begin{lstlisting}[language=SQL]
	-- Syntax:
	start transaction;
	-- execute DQL/DML commands
	savepoint pointA; -- pointA can be any valid name you wish
	-- execute DQL/DML commands
	savepoint pointB;
	-- execute DQL/DML commands
	-- ..
	rollback to savepointA; -- or to any other savepoint you wish to revert to
	-- this will result in changes done up to save pointA to be committed
	-- but any changes after pointA are reverted
	commit;
	
	-- Example:
	start transaction;
	
	select * from book where title = 'Experimental Cooking Recipes'; -- we have this book from before
	-- change publisher
	update book set release_year = 2014 where title = 'Experimental Cooking Recipes';
	select * from book where title = 'Experimental Cooking Recipes'; -- see updated release year
	savepoint pointUpdate;
	
	delete from book where title = 'Experimental Cooking Recipes'; -- book is deleted
	select * from book where title = 'Experimental Cooking Recipes'; -- no such book
	savepoint pointDelete;
	
	rollback to pointUpdate; -- keep changes up to that save point and revert anything done below that point
	commit;
	
	select * from book where title = 'Experimental Cooking Recipes'; -- book will be there with updated release year
\end{lstlisting}
\subsection{Access Modes: Read Write/Read Only}
\paragraph{} There are two transaction access modes: 1. READ WRITE (default) and READ ONLY. READ WRITE access mode allows you to execute both \acs{DQL} and \acs{DML} commands within a transaction while READ ONLY allows you to execute only \acs{DQL} queries. By default a transaction is in READ WRITE access mode.
\begin{lstlisting}[language=SQL]
	-- Syntax:
	start transaction read only;
	
	start transaction read write;
	start transaction; -- by default this will also get read write access mode
	
	-- Example
	start transaction read only;
	select * from book where title = 'Experimental Cooking Recipes'; -- book will be there with updated release year
	update book set release_year = 2013 where title = 'Experimental Cooking Recipes'; -- this or any other DML command (insert/update/delete) will fail
	commit;
	
	start transaction read write; -- or start transaction; has the same access mode
	select * from book where title = 'Experimental Cooking Recipes'; -- book will be there with updated release year
	update book set release_year = 2013 where title = 'Experimental Cooking Recipes'; -- now both commands will work
	commit;
\end{lstlisting}
\subsection{Intro to Transaction Isolation Levels}
\paragraph{} In below sections we will see in action the transaction isolation levels we saw in the previous section. To understand the effect of isolation levels we will be working on more than one session in below sections.\\\textbf{Note:} Transaction Isolation Level can only be set before transaction starts, not during.
\subsection{Practice: Transaction Isolation Levels - Read Uncommitted}
\begin{lstlisting}[language=SQL]
	-- Read uncommitted else dirty reads: this isolation level allows you to see changes that are not yet committed
	-- Lowest Isolation Level
	-- Effects: (1) Dirty Reads, (2) Non Repetable Reads (3) Phantom Reads
	-- Solves: -
	
	-- Session A: Start
	start transaction;
	select * from book where title = 'Experimental Cooking Recipes';
	-- change release year from 2013 to 2015
	update book set release_year = 2015 where title = 'Experimental Cooking Recipes';
	-- before commit - go to part Session B
	commit; -- once steps on Session B are done, do commit
	-- Session A: End
	
	-- Copy the content of Session B in another session
	-- Session B: Start
	set transaction isolation level read uncommitted;
	start transaction; -- now isolation level is set to read uncommitted
	-- release year will be 2015, even though change is not committed yet
	select * from book where title = 'Experimental Cooking Recipes';
	commit;
	-- Session B: End
\end{lstlisting}
\subsection{Practice: Transaction Isolation Levels - Read Committed}
\begin{lstlisting}[language=SQL]
	-- Read committed: this isolation level allows you to only see committed changes
	-- Second Lowest Isolation Level
	-- Effects: (1) Non Repeatable Reads (2) Phantom Reads
	-- Solves: (1) Dirty Reads (Uncommitted Changes)
	
	-- Non Repeatable Reads: Same query in one transaction will read different values due to changes made by a different transaction
	
	-- Session A.1: Start
	-- go to Session B.1
	-- Session A.1: End
	
	-- Session A.2: Start
	start transaction;
	update book set release_year = 2005 where title = 'Experimental Cooking Recipes';
	commit;
	-- go to Session B.2
	-- Session A.2: End
	
	-- Session A.3: Start
	start transaction;
	update book set release_year = 2035 where title = 'Experimental Cooking Recipes';
	commit;
	-- go to Session B.3
	-- Session A.3: End
	
	-- Copy the content of Session B in another session
	-- Session B: Start
	
	-- Session B.1: Start
	set transaction isolation level read committed; -- set transaction isolation level to read committed
	start transaction;
	select * from book where title = 'Experimental Cooking Recipes'; -- release year is 2015
	-- go to section A.2
	-- Session B.1: End
	
	-- Session B.2: Start
	select * from book where title = 'Experimental Cooking Recipes'; -- release year is 2005
	-- go to section A.3
	-- Session B.2: End
	
	-- Session B.3: Start
	select * from book where title = 'Experimental Cooking Recipes'; -- release year is 2035
	commit;
	-- Session B.3: End
	
	-- Notice that at each select in same transaction we read different value for release year.
	
	-- Session B: End
\end{lstlisting}
\subsection{Practice: Transaction Isolation Levels - Repeatable Read}
\begin{lstlisting}[language=SQL]
	-- Repeatable read: Guarantees that in one transaction, same query will return same
	-- values, even if they are updated by another transaction 
	-- Effects: (1) Phantom Reads
	-- Solves: (1) Dirty Reads (Uncommitted Changes) (2) Non Repeatable Reads
	
	-- Session A.1: Start
	-- go to Session B.1
	-- Session A.1: End
	
	-- Session A.2: Start
	start transaction;
	update book set release_year = 2008 where title = 'Experimental Cooking Recipes';
	commit;
	-- go to Session B.2
	-- Session A.2: End
	
	-- Copy the content of Session B in another session
	-- Session B: Start
	
	-- Session B.1: Start
	set transaction isolation level repeatable read; -- set transaction isolation level to repeatable read
	start transaction;
	select * from book where title = 'Experimental Cooking Recipes'; -- release year is 2035
	-- go to section A.2
	-- Session B.1: End
	
	-- Session B.2: Start
	select * from book where title = 'Experimental Cooking Recipes'; -- release year is still 2035
	commit;
	start transaction;
	select * from book where title = 'Experimental Cooking Recipes'; -- now new transaction sees updated value (release year 2008)
	commit;
	-- Session B.2: End
	
	-- Even though value is updated and committed at another transaction,
	-- this isolation level guarantees that results will be the same
	
	-- Session B: End
\end{lstlisting}
\subsection{Practice: Transaction Isolation Levels - Serializable}
\begin{lstlisting}[language=SQL]
	-- Serialable: Prevents other transactions from updating data this transaction has acquired lock on
	-- Highest Isolation Level
	-- Effects: -
	-- Solves: (1) Dirty Reads (Uncommitted Changes) (2) Non Repeatable Reads (3) Phantom Reads
	
	-- Phantom Reads: One transaction works on a set of data, that transaction keeps seeing exactly same data
	-- during its execution, however other transactions are allowed to change them (insert/delete/update). 
	-- Serializable prevents this.
	
	-- Note: If a transaction on Isolation Level: REPEATABLE READ makes change on some data (for example delete) then another transaction on Isolation Level: Serializable reads these data and then the previous transaction commits the change, the latest transaction still sees inconsistent data.
	
	-- Session A.1: Start
	-- go to Session B.1
	-- Session A.1: End
	
	-- Session A.2: Start
	start transaction;
	select * from book where release_year > 1980 and release_year < 1990; -- works
	update book set release_year = 18 where release_year = 1980; -- fails because table is locked
	delete from book where release_year = 1980; -- fails because table is locked
	insert into book (title, release_year, book_language) values ('New SQL Book', 2004, 'English'); -- fails again
	-- below works because only table book was used and thus locked by the other transaction
	select * from author where first_name in ('Virginia', 'John');
	insert into author (first_name, last_name, birthday) values ('John', 'Wick', '1891-04-14');
	update author set birthday = '1883-01-25' where first_name = 'Virginia' and last_name = 'Wolf';
	update author set birthday = '1882-01-25' where first_name = 'Virginia' and last_name = 'Wolf';
	delete from author where first_name in ('Virginia', 'John');
	commit;
	-- go to B.2
	-- Session A.2: End
	
	-- Copy the content of Session B in another session
	-- Session B: Start
	
	-- Session B.1: Start
	set transaction isolation level serializable; -- set transaction isolation level to serializable
	start transaction;
	select * from book where release_year > 1980 and release_year < 1990; -- works
	-- go to section A.2
	-- Session B.1: End
	
	-- Session B.2: Start
	commit;
	-- go again to A.2, now they will work again
	-- Session B.2: End
	
	-- Session B: End
	
	-- Note: The precise way of locking varies based on DBMS and its configuration.
\end{lstlisting}
\subsection{Transaction Management}
\begin{lstlisting}[language=SQL]
	-- View transactions
	-- MySQL Syntax:
	select	*
	from	performance_schema.events_transactions_current
	where	state not in ('COMMITTED', 'ROLLED BACK');
\end{lstlisting}
\subsection{Section Notes}
\begin{itemize}
	\item For all \acs{SQL}/\acs{TCL} commands of this section \href{file:./source-items/sql/5-sql-tcl.sql}{click here}.
	\item For more on \acs{SQL} \acs{TCL} \href{https://dev.mysql.com/doc/refman/8.0/en/sql-transactional-statements.html}{click here}.
\end{itemize}
