
\newpage
\section{External Resources}
\begin{itemize}
	\item \href{https://dev.mysql.com/doc/refman/8.0/en/}{MySQL Reference Manual}
	\item\href{https://en.wikipedia.org/wiki/Database}{\acl{DB} - Wikipedia}
	\item\href{https://en.wikipedia.org/wiki/Relational_database}{Relational \acl{DB} - Wikipedia}
	\item\href{https://en.wikipedia.org/wiki/SQL}{\acs{SQL} - Wikipedia}
	\item\href{https://en.wikipedia.org/wiki/Database#Database_management_system}{\acs{DBMS} - Wikipedia}
	\item\href{https://en.wikipedia.org/wiki/Relational_database#RDBMS}{\acs{RDBMS} - Wikipedia}
	\item\href{https://en.wikipedia.org/wiki/NoSQL}{No\acs{SQL} - Wikipedia}
	\item\href{https://dev.mysql.com/downloads/installer/}{Download My\acs{SQL} - MySQL Official Site}
\end{itemize}


% last section of the document - contains the list of used acronyms
%\newpage % acronyms should be on a new page
\section{Acronyms}
\begin{acronym}
	\acro{DB}{Database}
	\acro{DBMS}{Database Management System}
	\acro{RDBMS}{Relational Database Management System}
	\acro{SQL}{Structured Query Language}
	\acro{DDL}{Data Definition Language}
	\acro{DML}{Data Manipulation Language}
	\acro{DCL}{Data Control Language}
	\acro{TCL}{Transaction Control Language}
	\acro{DQL}{Data Query Language}
	\acro{ACID}{Atomicity, Consistency, Isolation, Durability}
	\acro{cmd}{Command aka Windows Command Prompt}
	\acro{CRUD}{Create-Read-Update-Delete}
	\acro{API}{Application Programming Interface}
	\acro{P.K.}{Primary Key}
	\acro{F.K.}{Foreign Key}
	\acro{e.g.}{latin: exempli gratia - english: for example}
\end{acronym}
